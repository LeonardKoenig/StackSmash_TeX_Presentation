\documentclass[aspectratio=169]{beamer}
\usetheme{metropolis}

\usepackage{polyglossia}
\usepackage{fontspec}
\defaultfontfeatures{Mapping=tex-text}

\usepackage[scale=1]{ccicons} % CC-BY-SA for theme
\usepackage{graphicx}
%\usepackage{minted}


\subject{Smashing The Stack For Fun And Profit}
\title{Smashing The Stack For Fun And Profit}
\subtitle{Artikel von Aleph One}
\author{Leonard König}

\usepackage{listings}

\institute{%
Freie Universität Berlin\vspace*{5pt}\\
\includegraphics[width=3cm]{Fub-logo_svg.png}
}
\date{}

%% misc
%\setbeamercovered{transparent}
\setbeamertemplate{navigation symbols}{}

\usepackage{xcolor}

\definecolor{mygreen}{rgb}{0,0.6,0}
\definecolor{mygray}{rgb}{0.5,0.5,0.5}
\definecolor{mymauve}{rgb}{0.58,0,0.82}
\definecolor{mygreenblue}{rgb}{0,0.6,0.4}

\lstdefinestyle{default}{%
  numbers=left,
  stepnumber=1,
  numberstyle=\tiny,
  numbersep=2pt,
  basicstyle=\scriptsize,
  backgroundcolor=\color{gray!10},
  commentstyle=\color{mygreenblue},
  keywordstyle=\color{blue},
  stringstyle=\color{mymauve},
  tabsize=4,
}
\lstdefinelanguage
   [x64]{Assembler}     % add a "x64" dialect of Assembler
   [x86masm]{Assembler} % based on the "x86masm" dialect
   % with these extra keywords:
   {morekeywords={CDQE,CQO,CMPSQ,CMPXCHG16B,JRCXZ,LODSQ,MOVSXD, %
                  POPFQ,PUSHFQ,SCASQ,STOSQ,IRETQ,RDTSCP,SWAPGS, %
                  rax,rdx,rcx,rbx,rsi,rdi,rsp,rbp, %
                  r8,r8d,r8w,r8b,r9,r9d,r9w,r9b}} % etc.

\lstset{language=[x64]Assembler}


\lstset{%
  style=default,
  language=[x64]Assembler,
}

\begin{document}

\maketitle

\begin{frame}{Timeline}
\setbeamertemplate{section in toc}[sections numbered]
\tableofcontents
\end{frame}

\section{Einführung}
\begin{frame}[fragile]{Code-Fixes}
\begin{center}
Meine Anpassungen an den Code:
\end{center}
\begin{columns}
\begin{column}{.5\textwidth}
  \lstinputlisting[firstline=7, lastline=8, language=c]{./c_src/inj32/ex3_phrack.c}
\end{column}
\begin{column}{.5\textwidth}
  \lstinputlisting[firstline=9, lastline=13, language=c]{./c_src/inj32/ex3.c}
\end{column}
\end{columns}
\vfill
\begin{itemize}
\item Offset-Adresse von \(12\) zu \(13\) geändert
\item Adress-Differenz von \(8\) auf \(10\)
\item Cleanup / explizites Type-Casting
\end{itemize}
\end{frame}

\begin{frame}{Differenz Sprung-Adresse}
\lstinputlisting[firstline=29, lastline=32]{./c_src/inj32/ex3.c-objdump}

Differenz eindeutig 10 (\( \mathtt{0x804843d} - \mathtt{0x8048447} = \mathtt{0xa} = 10\))
\end{frame}

\begin{frame}{Stack-Layout}
\lstinputlisting[firstline=1, lastline=7, language=]{./c_src/inj32/other.text}
\end{frame}

\begin{frame}[fragile]{GAS/AT\&T vs.\ intel}
\begin{columns}
\begin{column}{.5\textwidth}
  \lstinputlisting[firstline=51, lastline=65, firstnumber=1]{./c_src/inj32/ex3.c-objdump}
\end{column}%
\begin{column}{.5\textwidth}
  \lstinputlisting[firstline=69, lastline=83, firstnumber=1]{./c_src/inj32/ex3.c-objdump}
\end{column}
\end{columns}
\end{frame}

\section{Code}
\begin{frame}[fragile]{Erklärung des Codes}
\begin{columns}
\begin{column}{.5\textwidth}
  \vspace*{-12pt}
  \onslide<1->{%
    \lstinputlisting[firstline=88, lastline=89, firstnumber=1]{./c_src/inj32/ex3.c-objdump}
  }
  \vspace*{-12pt}
  \onslide<2->{%
    \lstinputlisting[firstline=90, lastline=91, firstnumber=last]{./c_src/inj32/ex3.c-objdump}
  }
  \vspace*{-12pt}
  \onslide<3->{%
    \lstinputlisting[firstline=92, lastline=93, firstnumber=last, firstnumber=last]{./c_src/inj32/ex3.c-objdump}
  }
  \vspace*{-12pt}
  \onslide<4->{%
    \lstinputlisting[firstline=94, lastline=94, firstnumber=last]{./c_src/inj32/ex3.c-objdump}
  }
  \vspace*{-12pt}
  \onslide<5->{%
    \lstinputlisting[firstline=95, lastline=96, firstnumber=last]{./c_src/inj32/ex3.c-objdump}
  }
  \vspace*{-12pt}
  \onslide<6->{%
    \lstinputlisting[firstline=99, lastline=100, firstnumber=last]{./c_src/inj32/ex3.c-objdump}
  }
  \vspace*{-12pt}
  \onslide<7->{%
    \lstinputlisting[firstline=101, lastline=102, firstnumber=last]{./c_src/inj32/ex3.c-objdump}
  }
  \vspace*{-12pt}
  \onslide<8->{%
    \lstinputlisting[firstline=103, lastline=106, firstnumber=last]{./c_src/inj32/ex3.c-objdump}
  }
\end{column}%
\hfill
\begin{column}{.5\textwidth}
  \onslide<1->{%
    \lstinputlisting[firstline=9,
      lastline=10,
      firstnumber=1,
    language=c]{./c_src/inj32/ex3.c}
  }
  \onslide<5->{%
    \lstinputlisting[firstline=12,
      lastline=13,
      firstnumber=last,
    language=c]{./c_src/inj32/ex3.c}
  }
\end{column}
\end{columns}
\vspace*{-30pt}
\end{frame}

\section{Anmerkungen}
\begin{frame}{Komisches Stack-Layout}
\lstinputlisting[firstline=1,
  lastline=7,
  language=]{./c_src/inj32/other.text}
\begin{enumerate}
\item<1-> Komisches Stack-Layout wg. \texttt{-O0}
\item<2-> Jedoch mit anderen Optimierungen wird \texttt{function()} komplett weg-optimiert
\item<3-> Müssten Code modifizieren, sodass er schwer weg-optimiert werden kann
\end{enumerate}
\end{frame}

\begin{frame}{64-Bit Version}
\lstinputlisting[firstline=1,
  lastline=17,
  language=c]{./c_src/inj64/ex3.c}
\end{frame}

\begin{frame}{Fragen?}
\center{{\LARGE Fragen?}}
\end{frame}

\begin{frame}[plain]{}
\begin{center}
{\large \textbf{Danke für's Zuhören!}}
\end{center}
\vfill
Phrack-Artikel: \url{phrack.org/issues/49/14.html}\\
\LaTeX -theme: \texttt{mtheme}: \url{github.com/matze/mtheme} \hfill \ccbysa
\end{frame}
\end{document}
