\def\my_aspectratio{169}
\def\mysize{\scriptsize}

%\def\my_aspectratio{43}
%\def\mysize{\tiny}
\documentclass[aspectratio=\my_aspectratio]{beamer}
\usetheme{metropolis}

\usepackage{xunicode}
\usepackage{xltxtra}

\usepackage{polyglossia}
\usepackage{fontspec}
\defaultfontfeatures{Mapping=tex-text}

\usepackage[scale=1]{ccicons} % CC-BY-SA for theme
\usepackage{graphicx}
\usepackage{minted}

\subject{Smashing The Stack For Fun And Profit}
\title{Smashing The Stack For Fun And Profit}
\subtitle{Artikel von Aleph One}
\def\my_author{Leonard König}
\author{\my_author}

\institute{
Freie Universtiät Berlin\vspace{5pt}\\
\includegraphics[height=1cm]{fu-berlin__Logo_RGB.jpg}
}
\date{}
%\date{\today}

%% misc
%\setbeamercovered{transparent} 
\setbeamertemplate{navigation symbols}{} 
\begin{document}

\maketitle

\begin{frame}{Timeline}
\setbeamertemplate{section in toc}[sections numbered]
\tableofcontents
\end{frame}

\section{Einführung}
\begin{frame}{Code-Fixes}
\begin{center}
Meine Anpassungen an den Code:
\end{center}
\begin{minipage}{.5\textwidth}
\inputminted[firstline=7, lastline=8, fontsize=\mysize, linenos]{c}{./c_src/inj32/ex3_phrack.c}
\end{minipage}%
%
\begin{minipage}{.5\textwidth}
\inputminted[firstline=9, lastline=13, fontsize=\tiny, linenos]{c}{./c_src/inj32/ex3.c}
\end{minipage}
\vfill
\begin{itemize}
\item Offset-Adresse von \(12\) zu \(13\) geändert
\item Adress-Differenz von \(8\) auf \(10\)
\item Cleanup / explizites Type-Casting
\end{itemize}
\end{frame}

\begin{frame}{Differenz Sprung-Adresse}
\inputminted[firstline=29, lastline=32, fontsize=\mysize, linenos]{c-objdump}{./c_src/inj32/ex3.c-objdump}

Differenz eindeutig 10 (\( \mathtt{0x804843d} - \mathtt{0x8048447} = \mathtt{0xa} = 10\))
\end{frame}

\begin{frame}{Stack-Layout}
\inputminted[firstline=1, lastline=7, fontsize=\scriptsize]{Text}{./c_src/inj32/other.text}
\end{frame}

\begin{frame}{GAS/AT\&T vs. intel}
\begin{minipage}{.5\textwidth}
\inputminted[firstline=51, lastline=65, firstnumber=1, fontsize=\scriptsize, linenos]{gas}{./c_src/inj32/ex3.c-objdump}
\end{minipage}%
\begin{minipage}{.5\textwidth}
\inputminted[firstline=69, lastline=83, firstnumber=1, fontsize=\scriptsize, linenos]{nasm}{./c_src/inj32/ex3.c-objdump}
\end{minipage}
\end{frame}

\section{Code}
\begin{frame}{Erklärung des Codes}
\begin{minipage}{.5\textwidth}
\vspace*{-12pt}
\onslide<1->{%
\inputminted[firstline=88, lastline=89, firstnumber=1, fontsize=\mysize, linenos]{gas}{./c_src/inj32/ex3.c-objdump}
}
\vspace*{-12pt}
\onslide<2->{%
\inputminted[firstline=90, lastline=91, firstnumber=last,fontsize=\mysize, linenos]{gas}{./c_src/inj32/ex3.c-objdump}
}
\vspace*{-12pt}
\onslide<3->{%
\inputminted[firstline=92, lastline=93, firstnumber=last, firstnumber=last,fontsize=\mysize, linenos]{gas}{./c_src/inj32/ex3.c-objdump}
}
\vspace*{-12pt}
\onslide<4->{%
\inputminted[firstline=94, lastline=94, firstnumber=last, fontsize=\mysize, linenos]{gas}{./c_src/inj32/ex3.c-objdump}
}
\vspace*{-12pt}
\onslide<5->{%
\inputminted[firstline=95, lastline=96, firstnumber=last, fontsize=\mysize, linenos]{gas}{./c_src/inj32/ex3.c-objdump}
}
\vspace*{-12pt}
\onslide<6->{%
\inputminted[firstline=99, lastline=100, firstnumber=last, fontsize=\mysize, linenos]{gas}{./c_src/inj32/ex3.c-objdump}
}
\vspace*{-12pt}
\onslide<7->{%
\inputminted[firstline=101, lastline=102, firstnumber=last, fontsize=\mysize, linenos]{gas}{./c_src/inj32/ex3.c-objdump}
}
\vspace*{-12pt}
\onslide<8->{%
\inputminted[firstline=103, lastline=106, firstnumber=last, fontsize=\mysize, linenos]{gas}{./c_src/inj32/ex3.c-objdump}
}
\end{minipage}%
\begin{minipage}{.5\textwidth}
\onslide<1->{%
\inputminted[firstline=9, lastline=10, firstnumber=1, fontsize=\mysize, linenos]{c}{./c_src/inj32/ex3.c}
}
\onslide<5->{%
\inputminted[firstline=12, lastline=13, firstnumber=last, fontsize=\mysize, linenos]{c}{./c_src/inj32/ex3.c}
}
\end{minipage}
\vspace*{-30pt}
\end{frame}

\section{Anmerkungen}
\begin{frame}{Komisches Stack-Layout}
\inputminted[firstline=1, lastline=7, fontsize=\scriptsize]{Text}{./c_src/inj32/other.text}
\begin{enumerate}
\item<1-> Komisches Stack-Layout wg. \texttt{-O0}
\item<2-> Jedoch mit anderen Optimierungen wird \texttt{function()} komplett weg-optimiert
\item<3-> Müssten Code modifizieren, sodass er schwer weg-optimiert werden kann
\end{enumerate}
\end{frame}

\begin{frame}{64-Bit Version}
\inputminted[firstline=1, lastline=17, fontsize=\scriptsize, linenos]{c}{./c_src/inj64/ex3.c}
\end{frame}

\begin{frame}{Fragen?}
\center{{\LARGE Fragen?}}
\end{frame}

\setbeamertemplate{headline}
{
  \leavevmode%
  \hbox{%
  \begin{beamercolorbox}[wd=\paperwidth,ht=12ex,dp=3.5ex]{secsubsec}%
    \raggedright
    \hspace*{2em}%
    {{\large\my_author}\hfill\includegraphics[height=1cm]{fu-berlin__Logo_RGB.jpg}}%
    \hspace*{2em}%
  \end{beamercolorbox}%
  }%
}

\begin{frame}
\begin{center}
{\large \textbf{Danke für's Zuhören!}}
\end{center}
\vfill
Phrack-Artikel: \url{phrack.org/issues/49/14.html}\\
Xe\LaTeX -theme: \texttt{mtheme}: \url{github.com/matze/mtheme} \hfill \ccbysa
\end{frame}
\end{document}